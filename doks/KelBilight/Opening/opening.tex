\documentclass[a4paper, 12pt]{article}
\usepackage[T2A]{fontenc}
\usepackage[left=2cm,right=2cm,top=2cm,bottom=2cm]{geometry}
\usepackage[russian]{babel}
\usepackage{amsfonts,amsmath,amssymb}
\usepackage{mathrsfs}
\usepackage{graphicx}
\usepackage[normalem]{ulem}
\usepackage{wrapfig}
\usepackage{fancyhdr}
\usepackage{floatflt}
\usepackage{python}
\usepackage{indentfirst}
\usepackage{setspace}
\usepackage{scrextend}
\usepackage{listings}
\usepackage{makecell,tabularx}
\usepackage{hyperref}
\usepackage{xcolor}

\newcommand{\rub}{{\rm{Р}\kern-.635em\rule[.5ex]{.52em}{.04em}\kern.11em}}

\definecolor{linkcolor}{HTML}{000000} 
\definecolor{urlcolor}{HTML}{0000FF} 

\hypersetup{pdfstartview=FitH,  linkcolor=linkcolor,urlcolor=urlcolor, colorlinks=true}

\definecolor{grey}{RGB}{40, 40, 40}

\renewcommand{\href}[1]{\url{#1}}

\lstdefinestyle{CommentStyle}{
    language=XML,
    %numbers=left, numberstyle=\tiny, stepnumber=1, numbersep=5pt,
    commentstyle=\color{red},
	basicstyle=\footnotesize\ttfamily,
	language={[ANSI]C++},
	keywordstyle=\bfseries,
	showstringspaces=false,
	morekeywords={include, printf},
	commentstyle={},
	escapeinside=§§,
	escapebegin=\begin{russian}\commentfont,
	escapeend=\end{russian},
    keywordstyle=\color{blue}\bfseries,
    morekeywords={align,begin},
    extendedchars=\true,
    tabsize=2
}
\lstdefinestyle{myLatexStyle}{
    language=c++,
    %backgroundcolor=\color{grey},
    numbers=left, numberstyle=\tiny, stepnumber=1, numbersep=5pt,
    commentstyle=\color{red},
    keywordstyle=\color{blue}\bfseries,
    morekeywords={align,begin},
    extendedchars=\true,
    tabsize=2
}

\lstdefinestyle{pmyLatexStyle}{
    language=java,
    %backgroundcolor=\color{grey},
    numbers=left, numberstyle=\tiny, stepnumber=1, numbersep=5pt,
    commentstyle=\color{red},
    keywordstyle=\color{blue}\bfseries,
    morekeywords={align,begin},
    extendedchars=\true,
    tabsize=2
}

\setlength{\parindent}{12,5mm}

\onehalfspacing

\pagestyle{fancy}
\renewcommand{\sectionmark}[1]{\markright{#1}}
\fancyhf{} 
\fancyhead[R]{\bfseries\thepage}
\fancyhead[LO]{\bfseries\rightmark}

\newcommand{\image}[3]{
	\begin{figure}[ht]
		\center{\includegraphics[height=#2pt]{img/#1} }
		\caption{\textit{#3}}\end{figure}
}                                                                                                                                             
\usepackage{tikz}                                                                                                          
\usetikzlibrary{calc}                                                                                                      
\usepackage{graphicx}                                                                                                      
\usepackage{newtxtext}                                                                                                     
\usepackage{float}                                                                                                         
\usepackage{comment}                                                                                                                
\newenvironment{myfont}{\fontfamily{phv}\selectfont}{\par}                       
\newcommand{\cmd}[1]{\immediate\write18{#1}}
\newcommand{\pf}[1]{\immediate\input{#1}}
\begin{document}
\thispagestyle{empty}
\begin{floatingfigure}{5.8cm}
  \includegraphics[width=5.8cm,height=3.5cm,keepaspectratio]{logo.png}
  \vspace{0cm}
\end{floatingfigure}

\sloppy{
  \scriptsize{
    \line(6,0){0}

    \centering Департамент образования и науки города Москвы

    \centering ГОСУДАРСТВЕННОЕ БЮДЖЕТНОЕ

    \centering ОБЩЕОБРАЗОВАТЕЛЬНОЕ УЧРЕЖДЕНИЕ ГОРОДА

    \centering МОСКВЫ «КУРЧАТОВСКАЯ ШКОЛА»

    \line(6,0){300}

    \centering 123060, Москва, улица Маршала Конева, дом 10.

    \centering \textbf{Тел: (499) 194-10-44, E-mail: kurchat@edu.mos.ru}

    \line(6,0){0}
  }}

\topskip=-200pt
\vspace*{130px}
\begin{Huge}
  \textbf{
    \begin{center}
      Проектная работа по теме:\\«Умное освещение для дома»
    \end{center}
  }
\end{Huge}
\vspace*{5px}
\begin{footnotesize}
  \center {В проекте был продемантсрирован прибор для освещения дома.}
  \vspace*{200px}
  \begin{flushright}
    Выполнил ученик 11 «А» класса\\
    Плютто Андрей Петрович\\
    Руководитель проекта\\
    ---------------------\\
  \end{flushright}
\end{footnotesize}
\begin{normalsize}
  \begin{center}
    Москва\\
    2021-2022 год
  \end{center}
\end{normalsize}
\newpage

\thispagestyle{fancy}
\renewcommand{\sectionmark}[1]{\markright{#1}}
\fancyhf{}
\fancyhead[R]{\bfseries\thepage}
\fancyhead[LO]{\bfseries Оглавление}
\renewcommand{\contentsname}{Оглавление}
\small{\tableofcontents}
\newpage
\pagestyle{fancy}
\renewcommand{\sectionmark}[1]{\markright{#1}}
\fancyhf{}
\fancyhead[R]{\bfseries\thepage}
\fancyhead[LO]{\bfseries\rightmark}

\section{Аннотация}

\newpage
\section{Введение}
\subsection{Актуальность}

%В наши дни тема умного дома набирает популярность и многие люди уже хотят себе
%установить пару-тройку модулей для упрощения повседневных дел. Многие 
%IT-гиганты бьются за место под солнцем, создавая новую или улучшая старую 
%технику до уровня "умного". Под влиянием этого я решил попробовать сделать 
%свой модуль этого же уровня для дополнительного освещения своей комнаты.

%Если говорить более подробно, то я решил сделать подсветку по периметру комнаты 
%с помощью адресной светодиодной ленты. Об актуальности данной идеи и говорить 
%не следует: в секторе освещения для умного дома выбора достаточно мало. Все, 
%что мне удалось найти на просторах интернета: пару устаревших модулей для 
%адресной светодиодной ленты под управлением с пульта и всевозможные виды умных
%ламп, которые, хоть и выглядят красиво, не производят такого эффекта как 
%лента. 

%В интернете я нашел проект, в котором взяли простую белую светодиодную ленту 
%и обклеили ей комнату. Получилось достаточно интересно. Различия между лентами
%я рассмотрел в %!

Тема умного дома набирает популярность в наши дни. Одно из главных направлений 
в этой области -- освещение. Не смотря на то, что это направление является 
очень интересным для покупателей многие компании не могут похвастаться большим
выбором. Я взял решил использовать адресную светодиодную ленту, так что 
вариантов ее свечения будет очень много. 

\subsection{Проблема}

Освещение комнаты при разном времени суток одинаково, что плохо влияет на 
зрение. Так же динамическое освещение хорошо влияет на настроение и 
психолгическое равновесие.

\subsection{Цель работы}

Создать динамическое освещение, подстраивающееся под уровень света в комнате и 
время суток.

\subsection{Задачи}
\begin{enumerate}
  \item Выяснить для разных естественных освещений какое должно быть искусственное
  \item Создать макет, развести и создать собственный МК для управления лентой
  \item Рассчитать потребление тока лентой 
  \item Создать программу для управление лентой
  \item Соединить МК и ленту
  \item Написать сайт для управления лентой с телефона 
\end{enumerate}
\\
\end{document}